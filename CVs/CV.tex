%%%%%%%%%%%%%%%%%%%%%%%%%%%%%%%%%%%%%%%%%%%%%%%%%%%%%%%%%%%%%%%%%%%%%%%%%%%%%%%%
% Medium Length Graduate Curriculum Vitae
% LaTeX Template
% Version 1.2 (3/28/15)
%
% This template has been downloaded from:
% http://www.LaTeXTemplates.com
%
% Original author:
% Rensselaer Polytechnic Institute
% (http://www.rpi.edu/dept/arc/training/latex/resumes/)
%
% Modified by:
% Daniel L Marks <xleafr@gmail.com> 3/28/2015
%
% Important note:
% This template requires the res.cls file to be in the same directory as the
% .tex file. The res.cls file provides the resume style used for structuring the
% document.
%
%%%%%%%%%%%%%%%%%%%%%%%%%%%%%%%%%%%%%%%%%%%%%%%%%%%%%%%%%%%%%%%%%%%%%%%%%%%%%%%%

%-------------------------------------------------------------------------------
%   PACKAGES AND OTHER DOCUMENT CONFIGURATIONS
%-------------------------------------------------------------------------------

%%%%%%%%%%%%%%%%%%%%%%%%%%%%%%%%%%%%%%%%%%%%%%%%%%%%%%%%%%%%%%%%%%%%%%%%%%%%%%%%
% You can have multiple style options the legal options ones are:
%
%   centered:   the name and address are centered at the top of the page
%               (default)
%
%   line:       the name is the left with a horizontal line then the address to
%               the right
%
%   overlapped: the section titles overlap the body text (default)
%
%   margin:     the section titles are to the left of the body text
%
%   11pt:       use 11 point fonts instead of 10 point fonts
%
%   12pt:       use 12 point fonts instead of 10 point fonts
%
%%%%%%%%%%%%%%%%%%%%%%%%%%%%%%%%%%%%%%%%%%%%%%%%%%%%%%%%%%%%%%%%%%%%%%%%%%%%%%%%

\documentclass[margin]{res}

% Default font is the helvetica postscript font
\usepackage{helvet}

% Increase text height
\textheight=700pt

\usepackage{graphicx}
\usepackage{multirow}

\begin{document}

%-------------------------------------------------------------------------------
%   NAME AND ADDRESS SECTION
%-------------------------------------------------------------------------------
\name{Rohit Chatterjee}

% Note that addresses can be used for other contact information:
% -phone numbers
% -email addresses
% -linked-in profile

% Simulate as if there are 7 lines of address
\address{\\\\\\\\Research Fellow\\ School of Computer Science \\National University of Singapore\\Singapore - 117417}
% Hence the photo would take 7 lines/rows
\address{\\\\\\\\rohitchatterjee94@gmail.com\\rochat@nus.edu.sg\\+65 8166 7354}

% Uncomment to add a third address
%\address{rohitchatterjee94@gmail.com\\Address 3 line 2\\Address 3 line 3}
%-------------------------------------------------------------------------------

\begin{resume}

%-------------------------------------------------------------------------------
%   POSITIONS SECTION
%-------------------------------------------------------------------------------
\section{POSITIONS}

Postdoctoral Research Fellow, \textbf{National University of Singapore}, Singapore
\\ Associated Faculty: Prashant N. Vasudevan \\\textit{November 2023 - Ongoing} 

%-------------------------------------------------------------------------------


%-------------------------------------------------------------------------------
%   EDUCATION SECTION
%-------------------------------------------------------------------------------
\section{EDUCATION}

\textbf{Stony Brook University}, Stony Brook, New York, USA\\
{\sl PhD Program}, Computer Science \& Engineering Department, September 2023 \\ Advisor: Prof. Omkant Pandey \\ \textit{Thesis:} Efficient Approaches to Emerging Cryptography
against Quantum Threats

\textbf{Indian Institute of Science}, Bangalore, Karnataka, India\\
{\sl Master of Science (Research)}, Undergraduate Department, 2017 (Math major)\hfill \newline
GPA: 5.8/8 \\
{\sl Bachelor of Science (Research)}, Undergraduate Department, 2016 (Math major)\hfill \newline
GPA: 6.1/8

%-------------------------------------------------------------------------------


%-------------------------------------------------------------------------------
%   PROJECTS SECTION
%-------------------------------------------------------------------------------

\section{PUBLICATIONS}

\employer{With Xiao Liang \& Omkant Pandey }
\location{}
\dates{{\it ICALP 2020}}
\title{\textbf{Improved Black-Box Constructions of Composable Secure Computation}}
\begin{position}
We construct a $\max(R_\mathsf{OT},\widetilde{O}(\log n))$-round  MPC protocol secure in the {\em angel-based} security model, by way of a constant-round black-box 1-1 CCA commitment scheme. The construction works under the modest assumption of semi-honest oblivious transfer. This closes the gap in round complexity between black-box and non-black-box MPC constructions in this model. 
\end{position}

\employer{With Sanjam Garg, Mohammad Hajiabadi, Dakshita Khurana, Xiao Liang, Giulio Malavolta, Omkant Pandey \& Sina Shiehian}
\location{}
\vspace{1mm}
\dates{{\it CRYPTO 2021}}
\title{\textbf{Compact Ring Signatures from Learning With Errors}}
\begin{position}
We present the first compact ring signature scheme (i.e., where the size of the signature grows logarithmically with the size of the ring) from the (plain) learning with errors (LWE) problem. The construction is in the standard model and it does not rely on a trusted setup or on the random oracle heuristic. At the heart of our scheme is a new construction of compact and statistically witness-indistinguishable ZAP arguments for $\mathsf{NP \cap coNP}$, that we show to be sound based on the plain LWE assumption. Prior to our work, statistical ZAPs (for all of NP) were known to exist only assuming sub-exponential hardness of LWE. 
\end{position}

\employer{With Kai-Min Chung, Xiao Liang, \& Giulio Malavolta}
\location{}
\vspace{1mm}
\dates{{\it PKC 2022}}
\title{\textbf{A Note on the Post-Quantum Security of (Ring) Signatures}}
\begin{position}
We consider signatures satisfying blind-unforgeability as recently proposed by Alagic et al. (Eurocrypt’20). We present two short signature schemes achieving this notion: one is in the quantum random oracle model, assuming quantum hardness of SIS; and the other is in the plain model, assuming quantum hardness of LWE with super-polynomial modulus. We further propose an analog of blind-unforgeability in the ring signature setting. Moreover, assuming the quantum hardness of LWE, we construct a compiler converting any blind-unforgeable (ordinary) signatures to a ring signature satisfying our definition.
\end{position}

\employer{With Ghada Almashaqbeh}
\location{}
\vspace{1mm}
\dates{{\it SECRYPT 2023}}
\title{\textbf{Building Unclonable Cryptography: A Tale of Two No-cloning Paradigms}}
\begin{position}
Unclonable cryptography builds primitives that enjoy some form of unclonability, which are impossible in the classical model as classical data is inherently clonable. Quantum computing, with its no-cloning principle, offers a solution. Very recently, an alternative no-cloning technology has been introduced [Eurocrypt'22], showing that unclonable polymers---proteins---can also be used to build bounded-query memory devices and unclonable cryptographic applications. In this work, we investigate the relation between these two technologies; towards this goal, we review the quantum and unclonable polymer models, discuss whether these primitives can be built using the other technology, and show alternative constructions and notions when possible. We also offer insights and remarks for the road ahead.
\end{position}

\employer{With Supartha Podder \& Srijita Kundu}
\location{}
\vspace{1mm}
\dates{{\it STOC 2025}}
\title{\textbf{On the Necessity of Uncloneable Proof and Advice States}}
\begin{position}
We initiate the study of languages that necessarily need uncloneable quantum proofs and advice. We define strictly uncloneable versions of the classes QMA, BQP/qpoly and FEQP/qpoly. These formalize the following: given any family of candidate proof or advice states, a polynomial-time cloning algorithm cannot act on it to produce states that are jointly usable by k separate polynomial-time verifiers, for arbitrary polynomial k. This is a stronger notion than those considered in previous works, which only required the existence of a single family of proof or advice states that are uncloneable. We show that in the quantum oracle model, there exist languages in strictly uncloneable QMA and strictly uncloneable BQP/qpoly. We also show without using any oracles that the language, used by Aaronson, Buhrman and Kretschmer (2024) to separate FEQP/qpoly and FBQP/poly, is in strictly uncloneable FEQP/qpoly.
\end{position}

\employer{With Xiao Liang, Omkant Pandey \& Takashi Yamakawa}
\location{}
\vspace{1mm}
\dates{{\it CRYPTO 2025}}
\title{\textbf{The Round Complexity of Black-Box Post-Quantum Secure Computation}}
\begin{position}
We study the round-complexity of secure multi-party computation (MPC) in the post-quantum regime where honest parties and communication channels are classical but the adversary can
be a quantum machine. Our focus is on the fully black-box setting where both the construction as well
as the security reduction are black-box in nature. First, we introduce the first blackbox construction for post-quantum MPC in polynomial rounds, from the minimal assumption of post-quantum semi-honest oblivious transfers. Second, we give the first black-box and constant-round construction in the multi-party setting. Our construction can be instantiated using various standard post-quantum primitives including lossy public-key encryption, linearly homomorphic public-key encryption, or dense cryptosystems. En route, we obtain a black-box and constant-round post-quantum commitment that achieves a
weaker version of the standard 1-many non-malleability, from the minimal assumption of post-quantum one-way functions. All of these results were previously open in the post-quantum setting.
\end{position}


\iffalse
\section{RELEVANT COURSEWORK}
\par
\textbf{Mathematics}:
Linear Algebra, Algebra, Real Analysis, Probability Theory, Combinatorics, Measure Theory, Ordinary Differential Equations, Complex Analysis, Commutative Algebra \& Galois Theory, Partial Differential Equations, Functional Analysis

\par
\textbf{Computer Science}:
Algorithms and Programming, Automata Theory And Computability, Computational Complexity Theory, Theoretical Foundations of Cryptography, Discrete Mathematics, Design and Analysis of Algorithms, Approximation Algorithms, Randomness in Cryptography, Fundamentals of Computer Networks, Data Science Fundamentals

\par
\textbf{Miscellaneous}:
Foundations of Data Sciences, Information Theory, Concentration Inequalities, Information \& Communication Complexity
\fi


\iffalse

\section{EVENTS \& CONFERENCES}

\textbf{Symposium on Learning, Algorithms and Complexity}, a National Mathematics Initiative (NMI) event organized at the Indian Institute of Science in January 2015.
\par

\textbf{INDOCRYPT 2015},  held at IISc, in December 2015.

\textbf{BITS 2016}, a workshop on information theory also attended by computer scientists, held at IIT Bombay and TIFR, in January 2016.

\textbf{CRYPTO 2018 \& 2019},  at USCB, in August 2018 \& 2019.

\textbf{New York Crypto Days}, several throughout 2017 - 2019.

\textbf{NYCAC 2017 \& 2018}, an annual theory day held at CUNY.

\textbf{Columbia/NYU Theory Day 2017 \& 2018}, an annual theory day held at Columbia University or NYU.

%-------------------------------------------------------------------------------

\fi



\iffalse

\section{PAST PROJECTS}

\employer{Prof. Ramesh Hariharan }
\location{Indian Institute of Science}
\dates{June 2014}
\title{\textbf{Satistfaction Of Boolean String Populations Recurrently Generated By an Underlying Boolean Function}}
\begin{position}
This project aimed to build upon a paper by Papadimitrou et al. which modelled sexual reproduction, by using a Boolean function to model the action of natural selection. We obtained positive results for some specific cases.
\end{position}

\employer{Prof. Bhavana Kanukurthi and Prof. Himanshu Tyagi }
\location{Indian Institute of Science}
\dates{Thesis Project (April 2016)}
\title{\textbf{Information Theoretic Secure Computation}}
\begin{position}
 Information theoretic proofs have been known for the fact that only a very trivial class of functions are securely computable, given no additional common resources to the parties. We looked at the problem of what other classes of functions become securely computable given additional common resources to parties.
\end{position}

\fi


%-------------------------------------------------------------------------------
%   COMPUTER SKILLS SECTION
%-------------------------------------------------------------------------------
%-------------------------------------------------------------------------------
\section{AWARDS AND ACHIEVEMENTS}

\textbf{Recipient of the prestigious Kishore Vaigyanik Protsahan Yojana (KVPY) fellowship}, a National Fellowship
in Basic Sciences, funded by the Department of Science and Technology (DST), Government of
India in 2012 for showing promise in research in basic science (All India Rank: 159).
\par

\section{OTHER QUALIFICATIONS}

\textbf{Programming Languages known}: Python, Java, C, R \\
\textbf{Courses TA'ed}: Modern Cryptography, Analysis of Algorithms, Foundations of Computer Science(UG)

%-------------------------------------------------------------------------------
%   EXPERIENCE SECTION
%-------------------------------------------------------------------------------
% Modify the format of each position
\begin{format}
\title{l}\employer{r}\\
\dates{l}\location{r}\\
\body\\
\end{format}
%-------------------------------------------------------------------------------



\iffalse
%-------------------------------------------------------------------------------
%   Interests
%-------------------------------------------------------------------------------
\section{RESEARCH INTERESTS}
Secure Computation, Cryptography, Algorithms, Computational Complexity, Communication Complexity, Information Theory.
%-------------------------------------------------------------------------------
\fi

\end{resume}
\end{document}
